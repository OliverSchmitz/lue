\documentclass[a4paper,12pt]{article}
\usepackage{authblk}
\usepackage{geometry}
\geometry{
    a4paper,
    total={160mm, 247mm},
}
\usepackage{graphicx}
\usepackage[hidelinks]{hyperref}
\usepackage[utf8]{inputenc}  % This file contains utf8 encoded text
\usepackage{listings}


\bibliographystyle{plain}


\begin{document}
\title{
    Land cover change model
}
\author[1,2]{Kor de Jong}
\affil[1]{%
    Department of Physical Geography\\
    Faculty of Geosciences\\
    Utrecht University\\
    Princetonlaan 8A\\
    3584 CB Utrecht\\
    The Netherlands}
\affil[2]{%
    Department of Information and Computing Sciences\\
    Faculty of Science\\
    Utrecht University\\
    Princetonplein 5\\
    3584 CC Utrecht\\
    The Netherlands}
\maketitle
\tableofcontents

\section{General}
\begin{itemize}
    \item Goal of the original PCRaster Land Use Change model (PLUC,
        \cite{verstegen:2012}, \cite{hilst:2012}, \cite{verstegen:2014})
        is to evaluate where bioenergy crops can be cultivated without
        entering into competition with other important land uses from
        a economic or sustainability point of view, now and in the near
        future when population and food intake per capita and thus arable
        land is likely to increase.
    \item Our goal is to develop a simplified PLUC model, using only
        local and focal operations, to be able to evaluate the performance
        and scalability of the new modelling framework algorithms which
        are based on the AMT approach.
\end{itemize}

Model through time where certain land-uses can be allocated. Other
land-uses will have to shrink because of this.
% - bioenery crop plantations
% - urban areas
% - crops...

\emph{Active change}: expansion or contraction of the total area of
certain land use is explicitly steered by certain drivers. Other land
use types can \emph{change passively}, by expansion of contraction of
an active land use type.

Land use change is steered by two factors:
\begin{itemize}
    \item Demand of population for food, non-food crops and wood
    \item Growth rate of yield, defined by agricultural and livestock
        productivity
\end{itemize}


\section{Model}
\subsection{General template}
\begin{lstlisting}
# Initialize the state at the start of the simulation
initialize_state()

for t in range(nr_time_steps):
    # Update the state for a single time step,
    # according to the relevant temporal processes
    simulate_new_state()
\end{lstlisting}


\subsection{Land Use Change}
\begin{lstlisting}
# Load the spatial distribution of land use at start location in time
# of the simulation
initialize_state()

for t in range(nr_time_steps):
    # Simulate the change in spatially distributed land use over a
    # time step
    simulate_new_state()
\end{lstlisting}


\begin{itemize}
    \item For each time step:
        \begin{itemize}
            \item For each (dynamic) land use type (order matters):
                \begin{itemize}
                    \item Calculate suitability map
                    \item While demand of land use is not fulfilled:
                        \begin{itemize}
                            \item Given demand and maximum yield, add
                                or remove land of this land
                                use type. Actual location of the
                                expansion/contraction of land use types
                                is determined by suitability factors.
                        \end{itemize}
                \end{itemize}
            \item Update statistics
            \item Output temporal state
        \end{itemize}
    \item Output statistics
\end{itemize}

Dynamic land use types PLUC:
\begin{itemize}
    \item cropland
    \item mosaic cropland-pasture (grazed grassland)
    \item mosaic cropland-grassland (not grazed)
    \item pasture
    \item forest
\end{itemize}

Demand $d_{n,t}$ (kg year$^{-1}$) for products from land use type $n$
at time step $t$ is:

\begin{equation}
    d_{n,t} = p_t \times i_{n,t} \times r_t
\end{equation}

Here, $p_t$ denotes the number of inhabitants
in at $t$, intake $i_{n,t}$ (kg caput$^{-1}$ year$^{-1}$) is the demand
per capita of products from land use type $n$ at $t$, and $r_t$ (-)
is the extent to which the food demands are met by domestic supply at $t$.
Demand is a scalar value, per country.

% Data requirements:
% For each t, for each n:
% - nr_inhabitants per country (time series)
% - demand per capita of products from land use type n (time series)
% - self-sufficiency ratio (time series)

Potential yield $p_{n,t}$ (kg km$^{-2}$ year$^{-1}$) is the yield of
products from land use type $n$ at $t$ if the cell would be occupied by
that land use type:

\begin{equation}
    p_{n,t} = m_{n,t} \times f_n
\end{equation}

Here, $m_{n,t}$ is the maximum possible product yield (kg km$^{-2}$
year$^{-1}$) of products from land use type $n$ at $t$ (can increase
through time). $f_n \in [0,1]$ is the actual fraction of this yield
that can be reached in a cell. Yield is a spatial raster.

% Data requirements:
% For each t, for each n:
% - maximum possible product yield (time series (per country?))
%
% For each n:
% - fraction of yield that can be reached per cell (raster)

Total product yield $y$ at $t$ of a land use type is calculated by summing
the actual yield $c_{n,t}$ which is the potential yield for cells that
are currently occupied by type $n$ at $t$, and zero for cells occupied
by other land use types.

\begin{equation}
    y_{n,t} = \sum (c_{n,t} \times a)
\end{equation}

Sum over the whole spatial field and $a$ is the cell size in km$^2$

For every land use type a total suitability map $s_{n,t}\in[0,1]$,
indicating the aggregated appropriateness of a given location for land
use $n$ at time step $t$, is computed from its suitability factors:

\begin{equation}
    s_{n,t} = \sum_{i=1}^{i=max_i} (w_{i,n} \times u_{i,n,t})
\end{equation}

\begin{equation}
    \sum_{i=1}^{i=max_i} w_{i,n} == 1
\end{equation}

% Data requirements:
% For each t, for each n:
% - data required for calculating all suitability factors, some of which
%     may be temporal.
%
% For each suitability factor of each n:
% - Factor for weighting the suitability factor

\begin{lstlisting}
for t in range(nr_time_steps):
    for n in range(nr_active_land_uses):

        demand_n_t = demand(n, t)
        yield_n_t = current_yield(n, t)

        if demand_n_t > yield_n_t:
            while demand_n_t > yield_n_t:
                convert cell with max (sn,t) to n
                update yield_n_t
        else if demand_n_t< yield_n_t:
            while demand_n_t< yield_n_t:
                convert cell with min (sn,t) to 99
                update yield_n_t
        else:
            do nothing
\end{lstlisting}


\subsection{Demo model}





\bibliography{bibliography}

\end{document}


% --------------------------------------------------------------------------------
% - jaar tijdstap, 30 jaar, groooot gebied
% - keep track of how long each cell has the same land-use kind
% - 100x100m
% 
% - Given
%     - Current distribution of land-use kinds
%     - Trend in development of {urban, cropland}
%         - Provides nr_cells_to_change into new land-use kind per year
% - For each year:
%     - Calculate suitability a of a cell for changing into {urban, cropland}
%         - Weighted average of some factors:
%             - a = (w1 * a1 + w2 * a2 + w3 * a3) / (w1 + w2 + w3)
%             - Urban:
%                 - Number of cells in neighborhood that are urban
%                 - Green space
%                 - ...
%             - Cropland:
%                 - Number of cells in neighborhood that are cropland
%                 - Forrest
%                 - ...
%     - average_suitability = nr_cells_to_change / nr_cells_available_to_change
%     - average_suitability = map_total(a_urban) / map_total(defined(a_urban))
%     - urban = cover(urban, a_urban > uniform(1))
% 
% Notes:
% - input: externe demand
% - suitabilities integreren in een suitability voor een land use type
% - gebruik demand om cellen te selecteren
% - order nodig om beste n cellen te selecteren 
% 
% Makkelijker:
% - suitability 0-1, kans dat ie switcht
% - random generator gebruiken
% 
% - Suitability op basis van kansen
% - Daaruit n cellen trekker (a > uniform)
% 
% - Preserve correctness of area!!! e.g.: Albers Equal Area projection
% --------------------------------------------------------------------------------


